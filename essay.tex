\documentclass[14pt, a4paper]{extarticle}
\usepackage[14pt]{extsizes}
\usepackage[utf8]{inputenc}
\usepackage[russian]{babel}
\usepackage{amsmath}
\usepackage{amsfonts}
\usepackage{graphicx}
\usepackage{amssymb}
\usepackage[left=2.5cm,right=1cm,top=2cm,bottom=2cm,bindingoffset=0cm]{geometry}
\usepackage{graphicx}
\usepackage[T2A]{fontenc} 
\usepackage{multicol}
\usepackage{caption}
\usepackage{subcaption}
\usepackage{textcomp} 
\usepackage{setspace} 
\usepackage{chngcntr}
\usepackage[small]{titlesec} 
\setlength{\parindent}{1.25cm}
\renewcommand{\baselinestretch}{1.0} 
\usepackage{cmap}
\AtBeginDocument{\let\textlabel\label}
\usepackage{hyperref}
\hypersetup{pdftex,colorlinks=true,linkcolor = black, citecolor = blue, backref=page}
\usepackage[all]{hypcap}

%\bibliographystyle{gost780u}
\bibliographystyle{unsrt}

%\counterwithin{equation}{section}
\renewcommand{\theequation}{\arabic{section}.\arabic{subsection}.\arabic{equation}}
\newcommand{\RNumb}[1]{\uppercase\expandafter{\romannumeral #1\relax}}
\begin{document}
	\begin{titlepage}
		\begin{center}
			\hfill \break
			МОСКОВСКИЙ ФИЗИКО-ТЕХНИЧЕСКИЙ ИНСТИТУТ\\ (ГОСУДАРСТВЕННЫЙ УНИВЕРСИТЕТ)\\
			\hfill \break
			\hfill \break
			\hfill \break
			\hfill \break
			\hfill \break
			Департамент философии\\
			\hfill \break
			\hfill \break
			РЕФЕРАТ ПО ИСТОРИИ НАУКИ\\
			\hfill \break
			\hfill \break
			\large{\textbf{История развития и проблемы биометрического распознавания человека}}\\
			\hfill \break		
		\end{center}
		
		\begin{center}
			\hfill \break
			\parbox{0.9\textwidth}
			{
				Аспирант~--- Соломатин Иван Андреевич \\
				Научный руководитель аспиранта \underline{\hspace{3cm}} д.т.н. Матвеев И.А. \\
				Преподаватель департамента философии~--- Фурсов А.А. \\
			}
		\end{center}
		\hfill \break
		\hfill \break
		\hfill \break
		\hfill \break
		\begin{center} Москва, 2018 
		\end{center}
		\thispagestyle{empty} 
	\end{titlepage}
	
\tableofcontents
\newpage

\section{Введение}
На протяжении всей истории человечества задача распознавания человека была актуальной задачей, однако в последние годы интерес к этой задаче многократно возрос. В основном, это вызвано появлением новых технических средств и всё более глубоким их проникновением в повседневную жизнь людей.

В настоящее время слово "биометрия"\ у всех на слуху: биометрическое распознавание используется в смартфонах \cite{odinokikh2018high, sezan2014user, hwang2009keystroke}, в банковском деле \cite{fatima2011banking, venkatraman2008biometrics}, в криминалистике \cite{tistarelli2014biometrics, bouchrika2011using}, а так же в ???. Однако, сама постановка задачи распознавания человека по каким-либо признакам имеет глубокие корни, уходящие далеко в прошлое.

На самом деле, любой человек, не задумываясь, решает эту задачу каждый день по многу раз, когда узнаёт своих родных, друзей, коллег. Чаще всего, при социальном взаимодействии, человек выстраивает диалог и предпринимает те или иные действия исходя из знания того, с кем именно он взаимодействует. Таким образом, данная задача лежит в основе социального взаимодействия людей и именно поэтому она настолько актуальна и обсуждаема.

В данной работе обсуждается история развития различных систем распознавания, а так же некоторые проблемы, связанные с подобными системами.

\section{Постановка задачи распознавания человека}
\section{История развития систем распознавания человека}
\section{Проблемы биометрического распознавания человека}
	
\newpage
\bibliography{bibliography}
	
\end{document}
